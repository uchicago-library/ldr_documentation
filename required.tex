% \UseRawInputEncoding
% To produce a PDF of this document: make mesoamerican.pdf
\documentclass[11pt]{article}
% \usepackage[utf8]{inputenc}
\usepackage[T1]{fontenc}
\usepackage{graphicx}
\usepackage{grffile}
\usepackage{longtable}
\usepackage{wrapfig}
\usepackage{rotating}
\usepackage[normalem]{ulem}
\usepackage{amsmath}
\usepackage{textcomp}
\usepackage{amssymb}
\usepackage{capt-of}
\usepackage{hyperref}

\usepackage{listings}
\lstset{
  escapeinside={(*@}{@*)},
  }

\lstset{literate=
  {á}{{\'a}}1 {é}{{\'e}}1 {í}{{\'i}}1 {ó}{{\'o}}1 {ú}{{\'u}}1
  {Á}{{\'A}}1 {É}{{\'E}}1 {Í}{{\'I}}1 {Ó}{{\'O}}1 {Ú}{{\'U}}1
  {à}{{\`a}}1 {è}{{\`e}}1 {ì}{{\`i}}1 {ò}{{\`o}}1 {ù}{{\`u}}1
  {À}{{\`A}}1 {È}{{\'E}}1 {Ì}{{\`I}}1 {Ò}{{\`O}}1 {Ù}{{\`U}}1
  {ä}{{\"a}}1 {ë}{{\"e}}1 {ï}{{\"i}}1 {ö}{{\"o}}1 {ü}{{\"u}}1
  {Ä}{{\"A}}1 {Ë}{{\"E}}1 {Ï}{{\"I}}1 {Ö}{{\"O}}1 {Ü}{{\"U}}1
  {â}{{\^a}}1 {ê}{{\^e}}1 {î}{{\^i}}1 {ô}{{\^o}}1 {û}{{\^u}}1
  {Â}{{\^A}}1 {Ê}{{\^E}}1 {Î}{{\^I}}1 {Ô}{{\^O}}1 {Û}{{\^U}}1
  {ã}{{\~a}}1 {ẽ}{{\~e}}1 {ĩ}{{\~i}}1 {õ}{{\~o}}1 {ũ}{{\~u}}1
  {Ã}{{\~A}}1 {Ẽ}{{\~E}}1 {Ĩ}{{\~I}}1 {Õ}{{\~O}}1 {Ũ}{{\~U}}1
  {œ}{{\oe}}1 {Œ}{{\OE}}1 {æ}{{\ae}}1 {Æ}{{\AE}}1 {ß}{{\ss}}1
  {ű}{{\H{u}}}1 {Ű}{{\H{U}}}1 {ő}{{\H{o}}}1 {Ő}{{\H{O}}}1
  {ç}{{\c c}}1 {Ç}{{\c C}}1 {ø}{{\o}}1 {å}{{\r a}}1 {Å}{{\r A}}1
  {€}{{\euro}}1 {£}{{\pounds}}1 {«}{{\guillemotleft}}1
  {»}{{\guillemotright}}1 {ñ}{{\~n}}1 {Ñ}{{\~N}}1 {¿}{{?`}}1 {¡}{{!`}}1 
  {'}{{'}}1
}

\date{2023-08-12}
\title{Europeana Data Model (EDM) Required ProvidedCHO Elements for use with University of Chicago Library Digital Collections}
\author{Charles Blair}
\hypersetup{
 pdfauthor={},
 pdftitle={},
 pdfkeywords={},
 pdfsubject={},
 pdfcreator={}, 
 pdftitle={},
 pdflang={English}}
\begin{document}

\maketitle

\section*{Required Elements}
Some of the following represents local usage in addition to what EDM mandates. For a fuller discussion, see Europeana Data Model (EDM) Template for use with University of Chicago Library Digital Collections.

\begin{itemize}

  \item \underline{dc:title}.

  \item \underline{dcterms:title}. Copy value of dc:title.

  \item \underline{dcterms:date}. ``Recommended practice is to express the date, date/time, or period of time according to ISO 8601-1 [ISO 8601-1] or a published profile of the ISO standard, such as the W3C Note on Date and Time Formats [W3CDTF] or the Extended Date/Time Format Specification [EDTF]. If the full date is unknown, month and year (YYYY-MM) or just year (YYYY) may be used. Date ranges may be specified using ISO 8601 period of time specification in which start and end dates are separated by a '/' (slash) character. Either the start or end date may be missing.'' (https://www.dublincore.org/specifications/dublin-core/dcmi-terms/\#http://purl.org/dc/terms/date)

  \item \underline{dcterms:identifier}. String-literal values for locally-created ARK identifiers are used, e.g., "https://n2t.net/ark:61001/z9m865s34c01". Note: dc:identifier may be used for any original identifiers occurring in provided metadata.

  \item \underline{dc:language} for TEXT objects and for other object types if there is a language aspect. Language codes conforming to ISO 639 are used as values.

  \item \underline{dcterms:language}. Copy value of dc:language.

  \item \underline{dcterms:spatial}. Values from Getty's Thesaurus of Geographical Names (TGN) are used.

  \item \underline{dcterms:type}. Values should be drawn from the ``DCMI Type Vocabulary''.\footnote{https://www.dublincore.org/specifications/dublin-core/dcmi-terms/\#section-7} These are Collection, Dataset, Event, Image, InteractiveResource, MovingImage, PhysicalObject, Service, Software, Sound, StillImage, Text. ``Note that digital representations of, or surrogates for, [PhysicalObject] should use Image, Text or one of the other types.''

  \item \underline{edm:type} must occur and must use one of the following values (in upper case): TEXT, IMAGE, SOUND, VIDEO or 3D. (EDM specification). Map values for dcterms:type as follows.

\begin{verbatim}
  Image : IMAGE
  MovingImage : VIDEO
  Sound : SOUND
  StillImage : IMAGE
  Text : TEXT
\end{verbatim}

\end{itemize}

Local usage also specifies the following:

\begin{itemize}
\item \underline{erc:who} The value is drawn from dc[terms]:creator, dc[terms]:contributor, or dc[terms]:publisher, in that order. If none of those have values, then use erc:who ``(:unkn) unknown''.
\item \underline{erc:what} The value is drawn from dc[terms]:title or dc[terms]:description, in that order. If dc[terms]:title is ``(:unkn) unknown'' and dc[terms]:description is not supplied, then use erc:what ``(:unkn) unknown''.
\item \underline{erc:when} The value is drawn from dcterms:date.
\item \underline{erc:where} The value is drawn from dcterms:identifier.
\end{itemize}

\end{document}
