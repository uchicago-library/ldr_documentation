% \UseRawInputEncoding
% To produce a PDF of this document: make mesoamerican.pdf
\documentclass[11pt]{article}
% \usepackage[utf8]{inputenc}
\usepackage[T1]{fontenc}
\usepackage{graphicx}
\usepackage{grffile}
\usepackage{longtable}
\usepackage{wrapfig}
\usepackage{rotating}
\usepackage[normalem]{ulem}
\usepackage{amsmath}
\usepackage{textcomp}
\usepackage{amssymb}
\usepackage{capt-of}
\usepackage{hyperref}

\usepackage{listings}
\lstset{
  escapeinside={(*@}{@*)},
  }

\lstset{literate=
  {á}{{\'a}}1 {é}{{\'e}}1 {í}{{\'i}}1 {ó}{{\'o}}1 {ú}{{\'u}}1
  {Á}{{\'A}}1 {É}{{\'E}}1 {Í}{{\'I}}1 {Ó}{{\'O}}1 {Ú}{{\'U}}1
  {à}{{\`a}}1 {è}{{\`e}}1 {ì}{{\`i}}1 {ò}{{\`o}}1 {ù}{{\`u}}1
  {À}{{\`A}}1 {È}{{\'E}}1 {Ì}{{\`I}}1 {Ò}{{\`O}}1 {Ù}{{\`U}}1
  {ä}{{\"a}}1 {ë}{{\"e}}1 {ï}{{\"i}}1 {ö}{{\"o}}1 {ü}{{\"u}}1
  {Ä}{{\"A}}1 {Ë}{{\"E}}1 {Ï}{{\"I}}1 {Ö}{{\"O}}1 {Ü}{{\"U}}1
  {â}{{\^a}}1 {ê}{{\^e}}1 {î}{{\^i}}1 {ô}{{\^o}}1 {û}{{\^u}}1
  {Â}{{\^A}}1 {Ê}{{\^E}}1 {Î}{{\^I}}1 {Ô}{{\^O}}1 {Û}{{\^U}}1
  {ã}{{\~a}}1 {ẽ}{{\~e}}1 {ĩ}{{\~i}}1 {õ}{{\~o}}1 {ũ}{{\~u}}1
  {Ã}{{\~A}}1 {Ẽ}{{\~E}}1 {Ĩ}{{\~I}}1 {Õ}{{\~O}}1 {Ũ}{{\~U}}1
  {œ}{{\oe}}1 {Œ}{{\OE}}1 {æ}{{\ae}}1 {Æ}{{\AE}}1 {ß}{{\ss}}1
  {ű}{{\H{u}}}1 {Ű}{{\H{U}}}1 {ő}{{\H{o}}}1 {Ő}{{\H{O}}}1
  {ç}{{\c c}}1 {Ç}{{\c C}}1 {ø}{{\o}}1 {å}{{\r a}}1 {Å}{{\r A}}1
  {€}{{\euro}}1 {£}{{\pounds}}1 {«}{{\guillemotleft}}1
  {»}{{\guillemotright}}1 {ñ}{{\~n}}1 {Ñ}{{\~N}}1 {¿}{{?`}}1 {¡}{{!`}}1 
  {'}{{'}}1
}

%%%
%%% Upload template.tex, template.ttl, and template.pdf to https://github.com/uchicago-library/ldr_documentation
%%%

\date{2023-03-17}
\title{Europeana Data Model (EDM) Template for use with University of Chicago Library Digital Collections}
\author{Charles Blair}
\hypersetup{
 pdfauthor={},
 pdftitle={},
 pdfkeywords={},
 pdfsubject={},
 pdfcreator={}, 
 pdftitle={},
 pdflang={English}}
\begin{document}

\maketitle

\section{Triple Types}
All triples should be typed. Choose from among these types.

\begin{lstlisting}
a ore:ResourceMap
a ore:Aggregation
a edm:ProvidedCHO
a ore:Proxy # optional
a edm:WebResource
a rdfs:Resource # This is the default type. Use it when
                  none of the preceding apply.
\end{lstlisting}

Of the above, a ore:Proxy is optional, but included if we are basing the descriptive metadata in a edm:ProvidedCHO on pre-existing descriptive metadata (e.g., a cataloging record).

\section{Required Elements}
In this section, mandatory Europeana elements are indicated by \textbf{boldface}. Mandatory Europeana elements governed by either/or or if/then statements are indicated by \textit{italics}. Mandatory elements required by local usage are \underline{underlined}.

The following is with reference to Definition of the Europeana Data Model v5.2.8 06/10/2017.\footnote{https://pro.europeana.eu/files/Europeana\_Professional/Share\_your\_data/
Technical\_requirements/EDM\_Documentation//EDM\_Definition\_v5.2.8\_102017.pdf}

\subsection{ore:ResourceMap}

The following are required for an edm:ResourceMap. The requirements are indicated by example.

\begin{verbatim}
<b9m865s34c01/rem>
dcterms:creator <https://dldc.lib.uchicago.edu/> ;
dcterms:created "2022-02-09T11:21:52-06:00"^^xsd:dateTime ;
dcterms:modified "2022-02-09T11:21:52-06:00"^^xsd:dateTime ;
ore:describes <b9m865s34c01/aggregation> ;
a ore:ResourceMap .
\end{verbatim}

Note that when an ore:ResourceMap is being created, the values of dcterms:created and dcterms:modified are the same.

\subsection{ore:Aggregation}

The following are mandatory Europeana elements for the \textit{ore:Aggregation}.

\begin{itemize}
\item \textbf{edm:aggregatedCHO}
\item \textbf{edm:dataProvider}
\item Either \textit{edm:isShownAt} or \textit{edm:isShownBy} (... ``using both is recommended!'').\footnote{https://pro.europeana.eu/files/Europeana\_Professional/Share\_your\_data/
Technical\_requirements/EDM\_Documentation/EDM\_Mapping\_Guidelines\_v2.4\_102017.pdf, section 4.3, p. 29.} If \textit{edm:isShownBy} is used, then \textit{edm:object} must be supplied. (See 3.2.27.)
\item \textbf{edm:provider}
\item \textbf{edm:rights}
\end{itemize}

The differences between edm:isShownAt and edm:isShownBy is explained as follows.

edm:isShownAt: ``An unambiguous URL reference to the digital object on the provider’s web
site in its full information context. ... This property will contain a URL that will be active in the Europeana
interface. It will lead users to the digital object displayed on the provider’s
web site in its full information context. Use edm:isShownAt if you display the
digital object with extra information (such as header, banner etc). Also use it
for digital objects embedded in HTML pages (even where the page is
extremely simple).'' \textit{Definition of the Europeana Data Model}, Section 3.2.21.Is Shown At.

edm:isShownBy: ``An unambiguous URL reference to the digital object on the provider’s web
site in the best available resolution/quality. ... This property will contain a URL that will be active in the Europeana
interface. It will lead users to the digital object on the provider’s website
where they can view or play it. The digital object needs to be directly
accessible by the URL and reasonably independent at that location. If the
URL includes short copyright information or minimal navigation tools it can
be entered in edm:isShownBy.'' \textit{Definition of the Europeana Data Model}, Section 3.2.22.Is Shown By.

The above suggests that a digital object displayed on web pages with full information context should be indicated by edm:isShownAt, while a digital object in the best resolution should be indicated by edm:isShownBy. By convention, we indicate the former by appending ``/website'' to the URL containing the ARK identifier, and the latter by appending ``/file.tif'', ``/file.wav'', etc. if the file is going to be retrieved from the OCFL hierarchy on disk; otherwise it should have its own ARK identifier, e.g., if it is going to be served by an external system such as Panopto. If edm:isShownBy is used then edm:object must also be used. ``This will often be the same URL as given in edm:isShownBy.'' \textit{Definition of the Europeana Data Model}, Section 3.2.27.Object.

% (Use \textbf{edm:ugc} is for crowd-sourced material; it is mandatory only if applicable.)

The following is an example of an ore:Aggregation.

\begin{verbatim}
<b9m865s34c01/aggregation>
edm:aggregatedCHO <b9m865s34c01> ;
edm:datasetName "CAILLA" ;
edm:dataProvider "University of Chicago Library" ;
edm:isShownAt <b9m865s34c01/website> ;
edm:isShownBy <c8m342c21s02> ;
edm:object <c8m342c21s02> ;
edm:provider "University of Chicago Library" ;
edm:rights <https://rightsstatements.org/vocab/NoC-US/1.0/> ;
ore:isDescribedBy <b9m865s34c01/rem> ;
a ore:Aggregation .
\end{verbatim}  

The ore:Aggregation may contain edm:hasView [link to edm:WebResource]. The example above contains an optional edm:datasetName predicate.

\subsection{edm:ProvidedCHO}
The following are mandatory Europeana elements for the \textit{edm:ProvidedCHO}.

\begin{itemize}

  \item Either \textit{dc:description} or \underline{\textit{dc:title}} or both. Local usage requires \underline{dc:title}.

  \item One or more the following: \textit{dc:coverage}, \underline{\textit{dcterms:spatial}}, \textit{dc:subject}, \textit{dcterms:temporal}, or \textit{dc:type}. Local usage requires \underline{dc:type}. Values should be drawn from the ``DCMI Type Vocabulary''.\footnote{https://www.dublincore.org/specifications/dublin-core/dcmi-terms/\#section-7} ``Note that digital representations of, or surrogates for, [physical] objects should use Image, Text or one of the other types.'' Local usage requires \underline{dcterms:spatial}. Values from Getty's Thesaurus of Geographical Names (TGN) are used.

  \item \textit{dc:language} for TEXT objects. It is strongly recommended for other object types if there is a language aspect. Language codes conforming to ISO 639 are used as values.

  \item \textbf{edm:type} must occur and must use one of the following values (in upper case): TEXT, IMAGE, SOUND, VIDEO or 3D. Map DCMI types to EDM types as follows: ``Text'' -> ``TEXT''; ``Image'' -> ``IMAGE''; ``Sound'' -> ''SOUND''; ``MovingImage'' -> ``VIDEO''. 
\end{itemize}

In addition to what is specified above, the following are also required by local usage for the \textit{edm:ProvidedCHO}.
\begin{itemize}
  \item \underline{dcterms:identifier}. String-literal values for locally-created ARK identifiers are used, e.g., "https://n2t.net/ark:61001/z9m865s34c01/website" (\underline{dc:identifier} may be used for any original identifiers occurring in provided metadata).

  \item \underline{dc:date}. String-literal values conforming to ISO 8601 or the Extended Date/Time Format Specification (EDTF)\footnote{https://www.loc.gov/standards/datetime/} are used in that order.

\end{itemize}

For the following, see ``DCMI Kernel Metadata Community: Kernel Metadata and Electronic Resource Citations (ERCs)'', Section 3, Kernel Stories.\footnote{https://www.dublincore.org/groups/kernel/spec/\#anchor3} They will support responses to queries about the metadata when the appropriate syntax is provided to an ARK resolver: ``ARK resolvers must support the `?info' inflection for requesting metadata.''\footnote{https://datatracker.ietf.org/doc/html/draft-kunze-ark-28\#section-1.2} See further in that document for the ?info inflection and the use of Electronic Resource Citations [ERCs].)

\begin{itemize}
\item \underline{erc:who} The value is drawn from dc[terms]:creator, dc[terms]:contributor, or dc[terms]:publisher, in that order. If none of those have values, then use erc:who ``(:unkn) unknown''.
\item \underline{erc:what} The value is drawn from dc:title or dc:description, in that order. If dc:title is ``(:unkn) unknown'' and dc:description is not supplied, then use erc:what ``(:unkn) unknown''.
\item \underline{erc:when} The value is drawn from dc:date.
\item \underline{erc:where} The value is drawn from dcterms:identifier.
\end{itemize}

Any values which cannot be assigned should be drawn from ``DCMI Kernel Metadata Community: Kernel Metadata and Electronic Resource Citations (ERCs)'', Section 9.3, Special Kernel Standardized Value Codes.\footnote{https://www.dublincore.org/groups/kernel/spec/\#anchor16}. For a local copy see ``Null Values'', https://dldc.lib.uchicago.edu/dl/nullValues.html.

The following is an example of an edm:ProvidedCHO. This example does not include any elements specific to a particular digital collection which may also be included in the edm:ProvidedCHO. It includes only those elements discussed in this section.

\begin{verbatim}
<b9m865s34c01>
dc:contributor "Hanks, William F." ;
dc:contributor "Unnamed contributor (male)" ;
dc:description "Contents: \"History of Cooperativa\"" ;
dc:language "spa" ;
dc:language "yua" ;
dc:title "Yucatec Maya field recordings (Hanks, 1979-1987) Yucatán Maya" ;
dcterms:identifier "https://n2t.net/ark:61001/z9m865s34c01" ;
dcterms:spatial "7005600" ;
dcterms:temporal "1979" ;
edm:type "SOUND" ;
erc:who "Hanks, William F., interviewer; Unnamed contributor (male), consultant" ;
erc:what "Yucatec Maya field recordings (Hanks, 1979-1987) Yucatán Maya" ;
erc:when "1979" ;
erc:where "https://n2t.net/ark:61001/z9m865s34c01" ;
a edm:ProvidedCHO.
\end{verbatim}

\subsection{ore:Proxy}

The following are required for an ore:Proxy, which is optional; it is used only if one is associating a file of original descriptive metadata, e.g., MARCXML or Dublin Core, with the aggregation. The requirements are indicated by example. 

\begin{verbatim}
<b9m865s34c01/file.dc.xml>
dc:format "application/xml" ;
ore:proxyFor <b2nw3wm8552h> ;
ore:proxyIn <b2nw3wm8552h/aggregation> ;
a ore:Proxy .
\end{verbatim}

Note that when an ore:ResourceMap is being created, the values of dcterms:created and dcterms:modified are the same.

\subsection{edm:WebResource}

The following are required for an edm:WebResource. The requirements are indicated by example. For \underline{dcterms:format}, string-literal values conforming to Internet Media Types (MIME) are used (http://www.iana.org/assignments/media-types/).

\begin{verbatim}
<b9m865s34c01/file.wav> 
dcterms:format "audio/x-wav" ;
dcterms:identifier "https://n2t.net/ark:61001/z9m865s34c01/file.wav" ;
premis:fixity [
  rdf:type <https://id.loc.gov/vocabulary/preservation/cryptographicHashFunctions/sha512>;
  rdf:value "10b1a2959f2229ab52e3873651c1d99036caa1e5d641e0bf257"] ;
premis:originalName "FieldRecording1996-2.wav" ;
premis:size 44694436 ;
a edm:WebResource .
\end{verbatim}

\subsection{rdfs:Resource}

If rdfs:Resource is used, the following indicates its format. The example is for OCR data for a page object in Campus Publications.

\begin{verbatim}
<b2nw3wm8552h/00000001/file.xml>
dc:format "application/xml" ;
a rdfs:Resource . # This is the default type. 
\end{verbatim}

\pagebreak[4]
\section{Example}
This example consists of the required elements plus those specific to a particular digital collection. If anything in this example contradicts anything in the above, the above remains authoritative.

\lstinputlisting[extendedchars=false,breaklines=true]{template.ttl}

\end{document}
